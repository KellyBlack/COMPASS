

%=========================================================================
% Start of the introduction to work energy integrals
%=========================================================================
\preClass{Vector Components}

\begin{problem}
\item A 2,000 N force is applied to a 500 kg block of sandstone.

  \includegraphics[width=10cm]{ink/week7/dragBlock}

  \begin{subproblem}
  \item If the angle, $\alpha$, is $\frac{\pi}{6}$ determine the
    $\vec{i}$ and $\vec{j}$ components of the force.
    \vfill
  \item If the angle, $\alpha$, is $\frac{\pi}{3}$ determine the
    $\vec{i}$ and $\vec{j}$ components of the force.
    \vfill
  \item If the angle, $\alpha$, is $\frac{2\pi}{3}$ determine the
    $\vec{i}$ and $\vec{j}$ components of the force.
    \vfill
  \end{subproblem}
\end{problem}


\actTitle{Non-Constant Forces}
\begin{problem}
\item A 1,800 N force is applied to a 500 kg block of sandstone. The
  coefficient of friction is 0.2.

  \includegraphics[width=8cm]{ink/week7/dragBlock}

  \begin{subproblem}
    \item Make a sketch of the free body diagram.
      \vspace{6em}
    \item Determine the work done if the block is dragged 10 m and the
      angle is $\alpha=\frac{\pi}{6}$.
      \vfill
    \item Determine the work done if the block is pulled for 20
      seconds and the angle is $\alpha=\frac{\pi}{3}$.  
      \vfill

    \item Determine the smallest angle which can be used and move the
      block.
      \vfill
  \end{subproblem}

  \clearpage

\item Suppose that as the block is pulled the sandstone grinds off on
  the bottom. It starts at a mass of 500 kg. For each meter is moves
  2\% of the sandstone is removed. Assume that the block starts at the
  origin, $x=0$ and assume that the angle is
  $\alpha=\frac{\pi}{6}$. The coefficient of friction is 0.2.

  \begin{subproblem}
  \item Determine the mass of the block given its position.
    \vfill
  \item Determine the amount of force necessary to move the block at a
    constant speed given the block's position.
    \vfill
  \item Make a sketch of the force as a function of the block's
    position.
    \sideNote{ Be sure to label your axes and label all important
      points on your graph.}

    \vfill
  \end{subproblem}

\end{problem}

\actTitle{Work and Riemann Sums}
\begin{problem}
\item The block used in the previous activity is examined, and we wish
  to find the amount of work necessary to pull the block at a constant
  speed over a distance of 100m.  Suppose that as the block is pulled
  the sandstone grinds off on the bottom. It starts at a mass of 500
  kg and is pulled at a constant velocity. For each meter is moves 2\%
  of the sandstone is removed. Assume that the block starts at the
  origin, $x=0$ and assume that the angle is
  $\alpha=\frac{\pi}{6}$. The coefficient of friction is 0.2.

  \includegraphics[width=8cm]{ink/week7/dragBlock}

  \begin{subproblem}
  \item Redraw the sketch of the force as a function of the block's
    position.
    \sideNote{ Be sure to label your axes and label all important
      points on your graph.}

    \vfill

  \item Break the distance in five equal lengths across the 100m.  If
    you assume that the force is constant over each interval how much
    work is required to move the block 100m?
    \sideNote{Use the force required at the start of each interval.}

    \vfill

    \clearpage

  \item Break the distance in ten equal lengths across the 100m.  If
    you assume that the force is constant over each interval how much
    work is required to move the block 100m?
    \sideNote{Use the force required at the start of each interval.}

    \vfill

  \item Break the distance in twenty equal lengths across the 100m.  If
    you assume that the force is constant over each interval how much
    work is required to move the block 100m?
    \sideNote{Use the force required at the start of each interval.}

    \vfill

  \item Briefly explain what is happening with respect to your
    approximation of the work. Is it getting bigger or smaller? What
    can you do to make the approximation better?

    \vspace{3em}

  \end{subproblem}
\end{problem}

\postClass

\begin{problem}
\item Briefly state two ideas from today's class.
  \begin{itemize}
  \item 
  \item 
  \end{itemize}
\item 
  \begin{subproblem}
    \item
  \end{subproblem}
\end{problem}





%%% Local Variables:
%%% mode: latex
%%% TeX-master: "labManual"
%%% End:

