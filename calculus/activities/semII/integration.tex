%=========================================================================
% Start of first day's activities
%=========================================================================
\preClass{Riemann Sums}

\begin{problem}
  \item ~
\end{problem}


\actTitle{Riemann Sums}

\begin{problem}
\item A 5m rod has a charge density given by 
  \begin{eqnarray*}
    \lambda_q & = & 5x~\frac{\mathrm{C}}{\mathrm{m}},
  \end{eqnarray*}
  where $x$ is between 0m and 5m. 
  \begin{subproblem}
  \item Make a rough sketch of the rod.
    \vspace{4em}
  \item Divide the rod into 5 equal parts, and determine the length of
    each part, and the coordinates for the endpoints.
    \vfill
  \item Determine the charge density at the left endpoint of each part
    of the rod.  
    \vfill
    \clearpage
  \item Determine an estimate for the total charge of each part
    assuming that the charge density is roughly constant over each
    part.
    \vfill
  \item Determine an estimate for the total charge in the rod.
    \vfill
  \end{subproblem}
\end{problem}

\postClass

\begin{problem}
\item Briefly state two ideas from today's class.
  \begin{itemize}
  \item 
  \item 
  \end{itemize}
\item 
  \begin{subproblem}
    \item
  \end{subproblem}
\end{problem}



%=========================================================================
% Start of second day's activities
%=========================================================================
\preClass{Integrals}

\begin{problem}
  \item Determine the value of the following integrals.
    \begin{subproblem}
    \item $\int^5_0 2x ~ dx$
      \vfill
    \item $\int^5_0 2x^2 ~ dx$
      \vfill
    \item $\int^5_0 2x^2 - 2x ~ dx$
      \vfill
    \end{subproblem}
\end{problem}


\actTitle{Integrals}

\begin{problem}
\item A 5m rod has a charge density given by 
  \begin{eqnarray*}
    \lambda_q & = & x e^{-x^2}~\frac{\mathrm{C}}{\mathrm{m}},
  \end{eqnarray*}
  where $x$ is between 0m and 5m. 
  \begin{subproblem}
  \item Make a rough sketch of the rod.
    \vspace{4em}
  \item Divide the rod into 5 equal parts, and determine the length of
    each part, and the coordinates for the endpoints.
    \vfill
  \item Determine the formula for the charge density on the left
    endpoint of each part from above.
    \vfill
    \clearpage
  \item Determine the sum using sigma notation for the estimate of the
    total charge in the rod.
    \vfill
  \item Express the limit of the sum as an integral.
    \vfill
  \item Determine the amount of charge in the rod.
    \vfill
  \end{subproblem}
\end{problem}

\postClass

\begin{problem}
\item Briefly state two ideas from today's class.
  \begin{itemize}
  \item 
  \item 
  \end{itemize}
\item 
  \begin{subproblem}
    \item
  \end{subproblem}
\end{problem}



%%% Local Variables:
%%% mode: latex
%%% TeX-master: t
%%% End:
