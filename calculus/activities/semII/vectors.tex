
%=========================================================================
% Start of first day of vectors in 3D
%=========================================================================
\preClass{Introduction to Vectors in Three Dimensions.}

\begin{problem}
\item Determine a way to express the point $P$ shown in the diagram below as the sum of
the $\vec{\imath}$, $\vec{\jmath}$, and $\vec{k}$ vectors.

  \tdplotsetmaincoords{60}{100}
  \begin{tikzpicture}[tdplot_main_coords]
    % Draw the coordinates
    \coordinate (O) at (0, 0, 0);
    \draw[->] (O) -- +(4, 0, 0) node[font = \small, pos = 1.1] {\(x\)} coordinate (X);
    \draw[->] (O) -- +(0, 4, 0) node[font = \small, pos = 1.1] {\(y\)} coordinate (Y);
    \draw[->] (O) -- +(0, 0, 4) node[font = \small, pos = 1.1] {\(z\)} coordinate (Z);
    %
    % Draw the unit vectors
    \draw[very thick,blue,->] (0,0,0) -- (1,0,0) node[font=\small,pos=1.7,anchor=south west] {$\hat{\imath}$};
    \draw[very thick,blue,->] (0,0,0) -- (0,1,0) node[font=\small,pos=0.9,anchor=south west] {$\hat{\jmath}$};
    \draw[very thick,blue,->] (0,0,0) -- (0,0,1) node[font=\small,pos=0.9,anchor=south east] {$\hat{k}$};
    %
    % Draw a point in space
    \node[draw=none,shape=circle,fill, inner sep=2pt] (d1) at (3,1,2){};  % circle
    \node[font=\small,anchor=south] at (d1) {P};  % label
    \draw[thin,red,dashed] (3,0,0) -- (3,1,0) node[font=\small,pos=0.5,anchor=north] {1};
    \draw[thin,red,dashed] (0,1,0) -- (3,1,0) node[font=\small,pos=0.25,anchor=north west] {3};
    \draw[thin,red,dashed] (3,1,0) -- (3,1,2) node[font=\small,pos=0.75,anchor=east] {2};
  \end{tikzpicture}

  \item Determine the distance between the point $P(2,3)$ and $Q(-1,5)$.

    \vfill


\end{problem}


\actTitle{Vectors in Three Dimensions}
\begin{problem}
\item Make a sketch of the coordinate axes in three dimensions.

  \vfill
  \begin{subproblem}
      \item Make sure to label the axes.
      \item Add the point $P(4,1,2)$ to your plot.
      \item Add the point $Q(1,-1,0)$ to your plot.
      \item Add the vector $PQ$ to your plot.
  \end{subproblem}
  \clearpage

\item A point, $P$, has coordinates $(x,y,z)$ as shown in the diagram below.
  The point $Q$ is directly beneath $P$ at $(x,y,0)$.

\tdplotsetmaincoords{60}{100}
\begin{tikzpicture}[tdplot_main_coords]
  % Draw the coordinates
  \coordinate (O) at (0, 0, 0);
  \draw[->] (O) -- +(4, 0, 0) node[font = \small, pos = 1.1] {$x$} coordinate (X);
  \draw[->] (O) -- +(0, 4, 0) node[font = \small, pos = 1.1] {$y$} coordinate (Y);
  \draw[->] (O) -- +(0, 0, 4) node[font = \small, pos = 1.1] {$z$} coordinate (Z);
  \node[font=\small,anchor=east] at (0,0,0) {$O$};
  %
  % Draw a point in space
  \draw[->] (O) -- +(3, 2, 2) node[font = \small, pos = 1.1] {} coordinate (P);
  \node[font=\small,anchor=south] at (P) {P};  % label
  \draw[thin,red,dashed]  (3,0,0) -- (3,2,0) node[font=\small,pos=0.5,anchor=north] {$x$};
  \draw[thin,red,dashed]  (0,2,0) -- (3,2,0) node[font=\small,pos=0.25,anchor=north west] {$y$};
  \draw[thin,red,dashed]  (3,2,0) -- (3,2,2) node[font=\small,pos=0.75,anchor=west] {$z$};
  \draw[thin,blue,dashed] (0,0,0) -- (3,2,0) node[font=\small,pos=0.5,anchor=west] {$h$};
  \draw[thin,blue] (2.8,1.8,0) -- (2.8,1.8,.4) -- (3,2,.4);
  \node[draw=none,shape=circle,fill, inner sep=2pt] (Q) at (3,2,0) {};
  \node[font=\small,anchor=north] at (Q) {Q};
\end{tikzpicture}

\begin{subproblem}
  \item Determine the distance, $h$, from the origin, $O$, to the point $Q$.
    \vfill
  \item Determine the distance between $Q$ and $P$.
    \vfill
  \item Determine the distance between $P$ and the origin. (Hint: Draw a side view of the triangle $OPQ$.
    \vfill
\end{subproblem}

\clearpage

\item Determine the distance between two points, $P(x_0,y_0,z_0)$ and $Q(x_1,y_1,z_1)$.
   (Hint: The distance between the points is the same as the length of the vector $\vec{PQ}$.)

   \vfill

 \item Determine the formula for the set of all points that are a distance of 2 m from the point $Q(2,-1,5)$.

  \vfill

\end{problem}


\postClass

\begin{problem}
\item Briefly state two ideas from today's class.
  \begin{itemize}
  \item
  \item
  \end{itemize}
\item
  \begin{subproblem}
    \item
  \end{subproblem}
\end{problem}


%=========================================================================
% Start of first day of lines and parameterization of a line
%=========================================================================
\preClass{Lines}

\begin{problem}
\item Determine a way to express the point $P$ shown in the diagram below as the sum of
the $\vec{\imath}$, $\vec{\jmath}$, and $\vec{k}$ vectors.

  \tdplotsetmaincoords{60}{100}
  \begin{tikzpicture}[tdplot_main_coords]
    % Draw the coordinates
    \coordinate (O) at (0, 0, 0);
    \draw[->] (O) -- +(4, 0, 0) node[font = \small, pos = 1.1] {\(x\)} coordinate (X);
    \draw[->] (O) -- +(0, 4, 0) node[font = \small, pos = 1.1] {\(y\)} coordinate (Y);
    \draw[->] (O) -- +(0, 0, 4) node[font = \small, pos = 1.1] {\(z\)} coordinate (Z);
    %
    % Draw the unit vectors
    \draw[very thick,blue,->] (0,0,0) -- (1,0,0) node[font=\small,pos=1.7,anchor=south west] {$\hat{\imath}$};
    \draw[very thick,blue,->] (0,0,0) -- (0,1,0) node[font=\small,pos=0.9,anchor=south west] {$\hat{\jmath}$};
    \draw[very thick,blue,->] (0,0,0) -- (0,0,1) node[font=\small,pos=0.9,anchor=south east] {$\hat{k}$};
    %
    % Draw a point in space
    \node[draw=none,shape=circle,fill, inner sep=2pt] (d1) at (3,1,2){};  % circle
    \node[font=\small,anchor=south] at (d1) {P};  % label
    \draw[thin,red,dashed] (3,0,0) -- (3,1,0) node[font=\small,pos=0.5,anchor=north] {1};
    \draw[thin,red,dashed] (0,1,0) -- (3,1,0) node[font=\small,pos=0.25,anchor=north west] {3};
    \draw[thin,red,dashed] (3,1,0) -- (3,1,2) node[font=\small,pos=0.75,anchor=east] {2};
  \end{tikzpicture}

  \item Determine the distance between the point $P(2,3)$ and $Q(-1,5)$.

    \vfill

  \clearpage

\item A turtle starts at the origin. It moves at a constant speed of 2
  meters per minute. It is facing in a direction $\frac{\pi}{4}$
  radians from the $x$-axis and is moving in the first
  quadrant. Determine its position at any time.

  \clearpage

\end{problem}


\actTitle{Parameterization of a Line}
\begin{problem}
\item Find the equation for a line that goes through the point
  $<1,5,-2>$  and is in the direction  $<4,-2,6>$.

  \vfill

\item Suppose that an object's position is given by the formula you
  had in the previous question. Determine its velocity.
  \sideNote{Determining velocity from position is no different in the
    context of vectors.}

  \vfill

\clearpage
\item Suppose that an object has a constant velocity given my
  $<4,-2,6>$ m/sec. If the object starts at the point $<1,5,-2>$
  determine its position at any time.

  \vfill

\end{problem}


\postClass

\begin{problem}
\item Briefly state two ideas from today's class.
  \begin{itemize}
  \item
  \item
  \end{itemize}
\item
  \begin{subproblem}
    \item
  \end{subproblem}
\end{problem}



%%% Local Variables:
%%% mode: latex
%%% TeX-master: "labManual"
%%% End:
