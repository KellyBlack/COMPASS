%=========================================================================
% Start of activity on simple harmonic motion.
%=========================================================================
\preClass{Derivatives of Basic Trigonometric Functions}

\begin{problem}
\item Determine the derivatives of the following functions:
  \begin{subproblem}
  \item $\sin(3t)$
    \vfill
  \item $\cos(8t)$
    \vfill
  \end{subproblem}
\item Determine the anti-derivatives of the following functions:
  \begin{subproblem}
  \item $\cos(3t)$
    \vfill
  \item $\sin(8t)$
    \vfill
  \end{subproblem}
\end{problem}


\actTitle{Anti-Derivatives of Trigonometric Functions}
\begin{problem}
\item An object has an acceleration given by
  \begin{eqnarray*}
    v(t) & = & 2\sin(5t) + 5e^{-2t},
  \end{eqnarray*}
  and the initial velocity is $v_0$, and the initial position is $x_0$.
  \begin{subproblem}
  \item Determine the position at any time.
    \vfill
  \item Describe the qualitative behaviour of the position. Does the position
    oscillate, grow, or decay? For what values of $v_0$ and $x_0$ do
    you see different behaviours?
    \vfill
  \end{subproblem}

  \clearpage

\item Determine the first and second derivatives of the following functions.
  \begin{subproblem}
  \item $8\sin(5t)$
    \vfill
  \item $23\cos(2t)$
    \vfill
  \item $8\sin(5t)+2\cos(5t)$
    \vfill
  \item $23\cos(2t)-14\sin(2t)$
    \vfill
  \end{subproblem}

\end{problem}

\actTitle{The Relationship Between Velocity and Time}
\begin{problem}
\item 
  \begin{subproblem}
    \item
  \end{subproblem}
\end{problem}

\postClass

\begin{problem}
\item Briefly state two ideas from today's class.
  \begin{itemize}
  \item 
  \item 
  \end{itemize}
\item 
  \begin{subproblem}
    \item
  \end{subproblem}
\end{problem}




%%% Local Variables:
%%% mode: latex
%%% TeX-master: "labManual"
%%% End:
