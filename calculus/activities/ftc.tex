%=========================================================================
% Start of first activity on the FTC
%=========================================================================
\preClass{Basic Kinematics}

\begin{problem}
\item The velocity of objects are given by the equations
  below. Determine the position of each object. Assume each object
  starts at $x=1$m
  \begin{subproblem}
  \item $v(t) = 3t - 5$.
    \vfill
  \item $v(t) = -2 t^3 + 10$.
    \vfill
  \end{subproblem}
\item The position of an object is given by the equations
  below. Determine the velocity at any time.
  \begin{subproblem}
  \item $x(t) = e^{3t} + 4t  - 5$.
    \vfill
  \item $x(t) = 2\cos(\pi t) + \ln(t+2)$.
    \vfill
  \end{subproblem}
\end{problem}


\actTitle{Change in Position as Area}
\begin{problem}
\item A 2 kg object starts from rest, and it can only move left or
  right. The positive direction is to the right, and the $\vec{i}$
  component of the velocity is given in the plot below.

  \scalebox{0.5}{\input{py/week8/piecewiseConstantVelI_week8day2.pgf}}

  \begin{subproblem}
    \item Make a sketch of the free body diagram.
      \vspace{4em}
    \item Determine the change in the position of the object after 4
      seconds.
      \vfill
  \end{subproblem}
  \clearpage

\item A 2 kg object starts from rest, and it can only move left or
  right. The positive direction is to the right, and the $\vec{i}$
  component of the velocity is given in the plot below.

  \scalebox{0.5}{\input{py/week8/piecewiseConstantVekII_week8day2.pgf}}

  \begin{subproblem}
    \item Make a sketch of the free body diagram.
      \vspace{4em}
    \item Determine the change in the position of the object after 4
      seconds.
      \vfill
  \end{subproblem}
  \clearpage

\item A 2 kg object starts from rest, and it can only move left or
  right. The positive direction is to the right, and the $\vec{i}$
  component of the velocity is given in the plot below.

  \scalebox{0.5}{\input{py/week8/changingVelGrid_week8day2.pgf}}

  \begin{subproblem}
    \item Make a sketch of the free body diagram.
      \vspace{4em}
    \item Make a sketch of the force in the $\vec{i}$ direction using
      the axes above.
    \item Determine the change in the position of the object after 4
      seconds by dividing up the time into four equally spaced
      intervals and assuming that the force is constant over each
      interval. Sketch the assumed, piecewise constant force, on the
      axes above.  
      \vfill
    \item Determine the change in position of the object after 4
      seconds by dividing up the time into eight equally spaced
      intervals and assuming that the force is constant over each
      interval. Sketch the assumed, piecewise constant force, on the
      axes above.
      \vfill
  \end{subproblem}
  \clearpage

\end{problem}

\actTitle{The Area Function}
\begin{problem}
\item The velocity of an object is 
  \begin{eqnarray*}
    \vec{v}(t) & = & \cos(2t) \vec{i}.
  \end{eqnarray*}
  \begin{subproblem}
  \item Make a sketch of the velocity.  \sideNote{ Be sure to label
      your axes and label all important points on your graph.}
    \vfill
  \item Express the change in the position as a definite integral.
    \vfill
  \item Determine the change in the position by solving the definite
    integral.
    \vfill
  \item Make a sketch of the change in position and briefly describe
    the change in position.
    \vfill
  \end{subproblem}
  \clearpage

\item The velocity of an object is 
  \begin{eqnarray*}
    \vec{v}(t) & = & \lp 5 - 5 e^{t/3} \rp \vec{i}.
  \end{eqnarray*}
  \begin{subproblem}
  \item Make a sketch of the velocity.  \sideNote{ Be sure to label
      your axes and label all important points on your graph.}
    \vfill
  \item Express the change in the position as a definite integral.
    \vfill
  \item Determine the change in the position by solving the definite
    integral.
    \vfill
  \end{subproblem}

\end{problem}

\postClass

\begin{problem}
\item Briefly state two ideas from today's class.
  \begin{itemize}
  \item 
  \item 
  \end{itemize}
\item 
  \begin{subproblem}
    \item
  \end{subproblem}
\end{problem}


%=========================================================================
% Start of activity on u substitution
%=========================================================================
\preClass{The Chain Rule}

\begin{problem}
\item Determine the derivatives of the following functions:
  \begin{subproblem}
  \item $f(t) = \cos(4t) + \ln(5t^2)$.
    \vfill
  \item $g(t) = \sin\lp e^{4t^5} \rp  + e^{\cos\lp 8t-5 \rp}$.
    \vfill
  \item $h(t) = \ln(4+\sin(2t) ) \cdot \cos \lp e^{2t^8} \rp$.
    \vfill
  \end{subproblem}
\end{problem}


\actTitle{$u$-Substitution}
\begin{problem}
\item Determine the derivatives of the following functions. In each
  case $u=u(t)$ is a function of $t$.
  \begin{subproblem}
    \item $f(t) = e^{u(t)}$
      \vfill
    \item $g(t) = \cos(3 u(t)) $
      \vfill
    \item $h(t) = \ln\lp e^{u(t)} + 5 \rp.$
      \vfill
  \end{subproblem}

  \clearpage

\item The derivative of a function is given. Determine the
  anti-derivative of each function. In each case assume that $u=u(t)$
  is a function of $t$.
  \begin{subproblem}
  \item $f'(t) = \sin(u(t)) u'(t)$
    \vfill
  \item $f'(t) = e^{u(t)} u'(t)$
    \vfill
  \item $f'(t) = \frac{1}{u(t)} \cdot u'(t)$
    \vfill
  \end{subproblem}
\end{problem}

\actTitle{Using $u$-substitution to Find Anti-Derivatives}
\begin{problem}
\item The force acting on the car is given below. In each case
  determine the change in the momentum of the car from $t=0$ to $t=2$.
  \begin{subproblem}
    \item $\vec{F} = \cos(\pi t) \vec{i}.$
      \vfill
    \item $\vec{F} = \lp 2 - 2 e^{t/9} \rp \vec{i}.$
      \vfill
    \item $\vec{F} = \frac{1}{t+1} \ln\lp t+1 \rp \vec{i}.$
      \vfill
  \end{subproblem}

  \clearpage

\item A bag of sand lies on the floor, and its initial mass is 50
  kg. For each meter it is dragged sand is spilled, and it loses
  roughly 10\% of its mass. The coefficient of friction between the
  bag and the floor is approximately $0.1$. A rope is attached to the
  bag, and the following force is applied to the bag
  \begin{eqnarray*}
    \vec{F}(t) & = & 10e^{-2t} \vec{i} + 20 e^{-2t} \vec{j}.
  \end{eqnarray*}
  How much work does the force exert on the bag?

  \vfill

\end{problem}

\postClass

\begin{problem}
\item Briefly state two ideas from today's class.
  \begin{itemize}
  \item 
  \item 
  \end{itemize}
\item 
  \begin{subproblem}
    \item
  \end{subproblem}
\end{problem}

%=========================================================================
% Start of activities on practice of the ftc
%=========================================================================
\preClass{Definite Integrals}

\begin{problem}
\item For each function and bound determine the areas under the curves
  geometrically as well as analytically.
  \begin{subproblem}
  \item $f(t)=t$ for $t=0$ to $t=1$.
    \vfill
  \item $f(t)=t+1$ for $t=0$ to $t=1$.
    \vfill
  \item $f(t)=-t$ for $t=0$ to $t=1$.
    \vfill
  \item $f(t)=3t-4$ for $t=0$ to $t=2$.
    \vfill
  \end{subproblem}
\end{problem}


\actTitle{Impulse as a Definite Integral}
\begin{problem}
\item An object has a mass of 3 kg. It initially starts at rest at the
  point $\vec{x}=1\vec{i}$. It is subject to a force given by
  $\vec{F}(t) = \cos(\pi t) \vec{i}$.
  \begin{subproblem}
    \item Make a sketch of the free body diagram.
      \vspace{5em}
    \item Use the impulse/momentum theorem to determine the change in
      momentum of the object from $t=0$ to any time $t=T$.
      \vfill
    \item Solve the previous relationship for the velocity at any time
      and determine the position of the object at any time.
      \vfill
  \end{subproblem}

  \clearpage

\item An object has a mass of 3 kg. It initially starts at rest at the
  point $\vec{x}=1\vec{i}$. It is subject to a force given by
  $\vec{F}(t) = \sin(\pi t)\vec{j}$.
  \begin{subproblem}
    \item Make a sketch of the free body diagram.
      \vspace{5em}
    \item Use the impulse/momentum theorem to determine the change in
      momentum of the object from $t=0$ to any time $t=T$.
      \vfill
    \item Solve the previous relationship for the velocity at any time
      and determine the position of the object at any time.
      \vfill
  \end{subproblem}

  \clearpage

\item An object has a mass of 3 kg. It initially starts at rest at the
  point $\vec{x}=1\vec{i}$. It is subject to a force given by
  $\vec{F}(t) = \cos(\pi t)\vec{i}+\sin(\pi t)\vec{j}$.
  \begin{subproblem}
    \item Make a sketch of the free body diagram.
      \vspace{5em}
    \item Use the impulse/momentum theorem to determine the change in
      momentum of the object from $t=0$ to any time $t=T$.
      \vfill
    \item Solve the previous relationship for the velocity at any time
      and determine the position of the object at any time.
      \vfill
  \end{subproblem}

\end{problem}

\actTitle{Work as a Definite Integral}
\begin{problem}
\item 
  \begin{subproblem}
    \item
  \end{subproblem}
\end{problem}

\postClass

\begin{problem}
\item Briefly state two ideas from today's class.
  \begin{itemize}
  \item 
  \item 
  \end{itemize}
\item 
  \begin{subproblem}
    \item
  \end{subproblem}
\end{problem}



%%% Local Variables:
%%% mode: latex
%%% TeX-master: "labManual"
%%% End:
