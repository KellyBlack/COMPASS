
\newcommand{\mlc}[1] {\texttt{>> #1 }}
\newcommand{\ttt}[1]{\texttt{#1}}

\section{MATLAB} 

\textbf{Note:} For a general introduction to Matlab see \textit{The
  Matlab User's Guide, Reference Manual}, or other \textit{Matlab}
manuals.  You can buy the student edition of Matlab.  It comes with a
good manual. \index{matlab}\\


If you are using Matlab in other classes, you may consider buying the 
\textbf{MATLAB Guide} by Desmond J. Higham and Nicholas J. Higham, published
by Siam.\\ 
  

\subsection{Basic Commands}

Here is some useful information about executing basic Matlab commands.

In the class room, after you get a computer and before you turn it on, make sure 
the power cord and network cable are connected.  To execute Matlab:

$\rightarrow$  Power up the computer.

\hspace{.2in}$\rightarrow$  Click on the START button

\hspace{.4in}$\rightarrow$  Select Programs

\hspace{.6in}$\rightarrow$  Select Matlab

\hspace{.8in}$\rightarrow$  Select Matlab

Matlab opens displaying three windows:
\begin{enumerate}
  \item The Launch Pad window
  \item The Command History window
  \item The Command window
\end{enumerate}

Initially the cursor will be in the command window at the Matlab command prompt
"$\ttt{>>}$".
\vspace{0.25in}

At the Matlab command prompt you may type any Matlab command.  Follow the
instructions below to execute some basic Matlab commands.
\vspace{0.25in}

\subsection{Basic operations} Addition, subtraction, and arrays.  The 
command 'Press Enter' is abbreviated by ENTER. \\
\vspace{1in}


\begin{tabular}{lll} 
  \ttt{>>} 3+5     &ENTER   &Add two numbers and display results. \\ 
  \ttt{>>} 4-6     &ENTER   &Subtract two numbers and display results. \\ 
  \ttt{>>} 4-6;    &ENTER   &Subtract two numbers and do NOT
  display results.\\
  \ttt{>>} a = 12/5; &ENTER   &Divide and store result in variable 'a'.\\
  \ttt{>>} a        &ENTER  &Display content of variable 'a'. \\
  \ttt{>>} 1:2:11   &ENTER  &Generate a list of values. \\
  \ttt{>>} b=1:0.1:2 &ENTER & Fill an array with values from 1 to
  2 in increments of 0.1.\\
  \ttt{>>} b          &ENTER & Display content of 'b'. \\
  \ttt{>>} who      &ENTER  &List current variables. \\
  \ttt{>>} whos     &ENTER  &List current variables and their size.\\
  \ttt{>>} d = 5:1:15 &ENTER &Fill array 'd'. \\
  \ttt{>>} b + d     &ENTER &Add two arrays. \\
  \ttt{>>} b .* d    &ENTER & Multiply two arrays component wise. \\
  \ttt{>>} b * d     &ENTER & \\ 
  \ttt{>>} g = [1 2; 3 4; 5 6] &ENTER &Define a two dimensional array. \\
  \ttt{>>} g2 = 2*g  &ENTER & Multiply array by a constant. \\
\end{tabular}

\subsection{Creating a function plot} ~ \\
\begin{tabular}{lll}
  \ttt{>>} \verb!x = -5:0.1:5;!  &ENTER  &Set up the independent variable. \\
  \ttt{>>} y = \verb!x.^2!; &ENTER  & Compute dependent variable or
  y-values . \\ 
  \ttt{>>} z = \verb!(1/8)*x.^3!;      &ENTER  & Compute values for
  a second function. \\ 
  \ttt{>>} plot(x,y);     &ENTER  & Plot x versus y. \\
  \ttt{>>} hold on;      &ENTER &Hold current graph . \\
  \ttt{>>} plot(x,z,'r'); &ENTER &Plot graph of 'z', in red, on same
  axis as 'y', \\
  \ttt{>>} title('y, z, versus x'); & ENTER & Set the title for the
  plot. \\
  \ttt{>>} hold;         &ENTER &Release current plot.  Next plot
  property is 'replace'. \\ 
\end{tabular}

\subsection{Sums and loops} ~ \\
\begin{tabular}{lll}
  \ttt{>>} u = [ 1 2 3 4 5]  &ENTER &Define the row vector 'u'. \\
  \ttt{>>} v = [ 6 7 8 9 10 ]' &ENTER &Define the column vector 'v'.\\
  \ttt{>>} w = (1:0.2:20)' &ENTER & Define the column vector 'w'. \\
  \ttt{>>} whos     &ENTER &Check the dimensions of the vectors. \\
  \ttt{>>} help sum &ENTER &  Example:  Get help with the
  SUM command. \\ 
  \ttt{>>} sum(v,1) &ENTER &Add values of  vector 'v'.\\
  \ttt{>>} sum(u,1) &ENTER &Observe results. \\ 
  \ttt{>>} sum(u,2) &ENTER & Add values of vector 'v'. \\
  \ttt{>>} \%  This is a comment line  &ENTER & \\  
  \ttt{>>} \%  Next we will try a for loop. &ENTER & \\
  \ttt{>>} clear x; &ENTER &  clear x-variable.  \\      
  \ttt{>>} n = 10;  &ENTER &Set up loop limit. \\
  \ttt{>>} for i = 1:n, x(i) = i+1, end  &ENTER  & \\
  \ttt{>>} x   &ENTER &Display content of vector 'x'. \\
  \ttt{>>} clear &ENTER & Clear all variables. \\
\end{tabular}

\subsection{Editing and running M-files} You can do many useful computations
working entirely at the Matlab command line, but soon you will find it helpful
to save a list of commands to an M-file.  M-file are the equivalents of programs
functions, subroutines, and procedures in other programming languages.

An M-file is a text file that has a \textbf{.m} filename extension and contains 
Matlab commands.

There are two types:
\begin{itemize}
  \item{\textbf{Script M-files}} (or command files) have no input or output
  arguments and operate on variables in the workspace.  A script enables you to
  store a sequence of commands that are to be used repeatedly or will be needed
  at some future time.
  
  \item{\textbf{Function M-files}} contain a function definition line and can
  accept input arguments and return output arguments and their internal
  variables are local to the function (unless declared global).
\end{itemize}

\begin{enumerate}
  \item To create an M-file click on the "new file" icon \textbf{or}\\
  $\rightarrow$ Select File
  
  \hspace{.2in}$\rightarrow$  Select New 
  
  \hspace{.4in}$\rightarrow$  Select M-file\\
  A text edit window will open.
  
  \item Enter the following Matlab commands to create a Matlab script.\\
  \verb!% Plot the function y=x^2.!\\
  \verb!% This is my first M-file experience!\\
  \ttt{x=-5:0.001:5;}\\
  \verb!y=x.^2;!\\
  \ttt{plot (x,y)}
  
  \item Save the M-file to the E: drive in the "Calculus" folder as test1.m.
  
  If this is the first time you are running Matlab, you may need to create the
  folder "Calculus."
  
  When Matlab opens, the default directory is S:$\backslash$NetApps$\backslash$ 
  Matlab$\backslash$bin$\backslash$win32.  To change the directory click on the
  icon to the right of the displayed directory name.  The "Browse for Folder"
  window will open.\\
  $\rightarrow$ Click on E: to select the E: drive.\\
  $\rightarrow$ Click on the "New Folder" icon and type the word "Calculus" to
  name the new folder.\\
  $\rightarrow$ Now save your M-file to the folder "Calculus."
  
  \item To execute the Matlab script click on the "Save and Run" or "Run" icon. 
  The first time you select the "Run" command the Matlab Editor window will open
  containing the following information:
  
  File E:$\backslash$Calculus$\backslash$test1.m is not found in the current
  or on the MATLAB path.\\ 
  To run this file, select one of the following:
  \begin{itemize}
    \item Change MATLAB current directory
    \item Add directory to the top of the MATLAB path
    \item Add directory to the bottom of the MATLAB path.
  \end{itemize}
  Select "Add directory to the top of the MATLAB path" and click on OK.
  
  Matlab will then execute the script.  What do you observe?
  
  \item Next create a function called f1.m that will generate the points to plot
  the function $f(x)=e^{(-x/3)}sin(2x)$ using M-file.
  
  Open a new M-file and type the following commands:
  \begin{verbatim}
    function y=f1(x)
    % y=f1(x) : This is my second M-file experience
    y=exp(-x/3).*sin(2*x);
  \end{verbatim}
  Save the file as f1.m to the Calculus folder on the E: drive.
  
  To execute the function we first need to define $x$.  Return to the command
  window and define $x$  and execute the function as follow:\\
  \begin{verbatim}
    >> x1=-5:0.01:5;
    >> y1=f1(x1); % execute function
    >> plot(x1,y1)
  \end{verbatim}
  
  Matlab has a help command that can be invoked.  The Matlab help function outputs
  all the comment lines before the first blank line in an M-file.  Enter\\
  \mlc{help f1}\\
  The last example was a function with a single input and one output.  A function
  can include more than one input or output.  The next example will construct a
  function with one input and two outputs.
  
  Open a new M-file and type the following commands:
  \begin{verbatim}
    function [y,z]=f2(x)
    %[y,z]=f2(x)  My third M-file.
    y=exp(-x/3).*sin(2*x);
    z=exp(-x/3).*cos(2*x);
  \end{verbatim}
  Save the file as f2.m to the Calculus folder on the E: drive.
  
  To execute this function and to plot the output type the following commands in 
  the command window:\\
  \mlc{[y1,z1]=f2(x1);}\\
  \mlc{plot(x1,[y1,z1])}\\
  
  \item Our next example is a more complex M-file script.  Functions are like
  subroutines in that they have their own internal variables and communicate
  with the command window and the workspace through their inputs and outputs. 
  Scripts, on the other hand, are simply sequences of commands which operate on
  the variables in the workspace.  Any variable created in a script is placed in
  the workspace.  For that reason it is good practice to clear any variables
  which are no longer needed.  Scripts do not use inputs or create outputs and
  will give an error if you ask for it.  (Functions need not have outputs
  however.)  We will call the script f2\_plot.m.
  
  Open a new M-file and type the following commands:
  \begin{verbatim}
    % f2_plot	(Note: Not a "function" so first line can be a comment)
    % Script to plot output of function f2    
    [u,v] = f2(x1);
    figure(1)
    subplot(2,1,1); plot(x1,[u,v]);
    title('u, v versus x');
    xlabel('x1');
    ylabel('Feet');
    w = u.*v;
    z = u.^2+v.^2;
    subplot(2,1,2); plot(x1,[w,z]);
    title('w, z versus x');
    xlabel('x1');
    ylabel('Sq. Feet');
    clear u v w z
  \end{verbatim}
  Save the M-file as f2\_plot.m to the Calculus folder on the E: drive.
  
  To execute the M-file script either click on the "Run" icon or type f2\_plot
  at the \ttt{>>} command prompt in the command window.
  
  What do you observe?  What effect does the subplot command have?
  \vspace{.25in}
  
  \item We will do one more example of a function.  This function will create a
  sequence of numbers, called the Fibonacci sequence.  It will have one input
  and one output and use a "for" loop to generate the sequence.
  \clearpage
  Open a new M-file type the following commands including the comments:
  \begin{verbatim}
    function y=fibonacci(n)
    % y=fibonacci(n)
    % Populate the vector y with the first n terms of the Fibonacci sequence,
    % using a "for" loop.
    % The sequence is named after the Italian mathematician Leonardo Pisano,
    % nicknamed Leonardo Fibonacci, who lived about 1175-1250 AD.
    % After the second term, the Fibonacci sequence is generated by adding the
    two previous terms to generate the current term.
    y(1)=1;
    y(2)=1;
    for i=3:n
        y(i)=y(i-1)+y(i-2);
    end
  \end{verbatim}
  Save the M-file as fibonacci.m in the Calculus folder on the E: drive.
  
  To display the comments at the beginning of the fibonacci.m file and to
  execute it, type the following commands in the command window:
  \begin{verbatim}
  >>help fibonacci
  >>n=5
  >>y1=fibonacci(n)
  \end{verbatim}
  
  If you would like to plot the sequence type the following commands in the
  command window:
  \begin{verbatim}
  >>n=1:1:12;
  >>n1=12;
  >>y=fibonacci(n1);
  >>plot(n,y,'*')
  \end{verbatim}
  What do you observe?
  
     
\end{enumerate}



\section{Assignment Guidelines}


\subsection{Home Work} \index{Home Work}
All work that is handed in for a grade must be neat. Any work that
cannot be easily read will not be graded. Every home work assignment
must have your name at the top of every page and all sheets must be
stapled together. Assignments that are not stapled will have points
deducted.

The front page must have your name written on it. The front page must
also have three other items. First, the current assignment should be
written. Secondly, it should have the times that you planned in
advance to work on the homework. Third, it should have the times that
you did the actual work.

Each page must be one column. Your solutions to the home work problems
should be written out one after the other and not on the same line.
For example, after completing a problem the next problem should appear
directly below the previous problem. Once you reach the bottom of the
page you must move on to the next page and not start over in a second
column on the current page.

A front page should look like the following: \\
\begin{center}
\fbox{
\begin{minipage}[h]{4in}
Stuart Dent \\
Page 32 1-5 odd, 27, 29, and 35. \\
Page 107 21-31 odd, 35a, and 42. 

\vspace{2em}

Original times: Tuesday 5-6pm, Wednesday 4:30-5:30, and Thursday
6-7pm.

\vspace{2em}

Actual times: Thursday 8pm to Friday 2am.

\vspace{2em}

Page 32: \\
1) $\frac{d}{dt} t^2 = 2t$.

3) $\frac{d}{dt} \left( \sqrt{t} + t \right)  = 
    \frac{1}{2} t^{-\frac{1}{2}} + 1$

$\vdots$

\end{minipage}}
\end{center}

\subsection{Group Work} \index{Group Work}
For many of the assignments in this class you will be required to work
in small groups.  Please keep in mind the following guidelines when
taking part in group work: \index{group work}
\begin{enumerate}
\item Cooperate with each other. Work together and find a solution to
  the problem together.

\item Make sure that everyone understands each answer before the group
  moves on to the next question.

\item Listen carefully to each other and try, whenever possible, to
  build upon their ideas.

\item Share in the leadership of the group.  You may wish to assign
  different roles, i.e. moderator, recorder, etc., to different
  individuals.
  
\item Make sure that everyone participates and that no one dominates.
  If you see someone in the group who is lost, urge him or her to ask
  questions, or ask a simple question yourself to bring others into
  the discussion.

\item Ask questions when you don't understand something.

\item Do not be critical of other people's \textbf{questions}.
\end{enumerate}



\subsection{Projects} \index{Projects}
Formal projects will be assigned in this class. You will have between
one and two weeks to complete each project. You will either present
the project in a presentation in front of the class or turn in a
written report. 

Before the due date of the project you must meet with your professor
at least twice. All members of your group should be present at the
meeting. Meetings should be scheduled at the beginning of project work
on the Friday that projects are handed out.  \index{meeting} Your
group needs to meet for at least an hour before this meeting. The
total length of the meeting is 30 minutes. After 30 minutes the
meeting is concluded.

\subsubsection {Rewrites}
Written submissions can be rewritten and resubmitted within a week
after they are turned back to you.  \index{written projects} You may be
able to increase your grade significantly with a modicum of extra work
by using our suggestions to improve the quality of the presentation.


\subsubsection{Outline for the Written  or Oral Presentation}

The formal write up for this lab should be about five pages long and
should be typed. \index{written projects} But we will be more
concerned about whether or not the key questions are answered rather
than the length of the write up.  The text should be double spaced. Be
sure that {\it every} member of the group signs the lab report.  By
signing your name, you are indicating that you contributed
significantly toward the completion of that report.


Your report should include the topics given in the outline below.  You
do not have to follow the outline, but your report should address all
of the issues listed in the outline.  (For some reports the sample
outline may not be appropriate!)

%\begin{table}[hb]
\begin{itemize}
\index{written projects}
\item Title Page -- should include the following (in order):
  \begin{itemize}
  \item Title
  \item Date
  \item The typed name and signature of each member (on separate
    lines).
  \end{itemize}
\item Abstract -- A brief summary of the project
\item Introduction
  \begin{itemize}
  \item Very briefly describe the situation, finer details are given
     later.  Restate the problem in a way that points to a solution method.
     %(For example, Can we find reasonable and believable accelerations
     %and wait times for Joe and his mom that give Joe's travel time to
     %be 5 seconds less than his mom without Joe speeding.)
   \item Explicitly state your findings.  (You are expected to justify
     these findings in the sections that follow.)
  \end{itemize}

\item Physical Situation
  \begin{itemize}
  \item Include a full description of the physical system.
  \item Carefully define all of the important qualitative aspects.
  \item Include appropriate visualizations of the problem (e.g. graphs,
        free body diagrams).
  \item Be very careful about the flow of this part of your report. If
        the transitions are awkward we will take off points!
  \end{itemize}

\item Mathematical Model
  \begin{itemize}
  \item Describe and state the problem in mathematical terms.  Justify
    your plan of attack.
  \item  Discuss and justify any assumptions made.
  \item Include any other important definitions.
  \item  Present the details of your calculations.  Discuss clearly what you
    are doing at each step and why.
  \item Support your conclusions! A statement which is not carefully
    and correctly explained will not receive credit.
  \item Provide appropriate and convincing checks.
  \end{itemize}

\item Conclusion
  \begin{itemize}
  \item Include a brief description of the physical system.
  \item Briefly describe methods used to obtain result.
  \item Restate your findings and give a brief description of what is
    happening.
  \end{itemize}


\end{itemize}

%\caption{Sample Outline.}
%\label{sampleOutline}
%\end{table}


\subsubsection{Grade}

Your grade will be determined by the content and mechanics of your
final report. \index{grading} The only way that we have to determine
whether or not you understand the concepts discussed in class is
through the way in which you communicate your understanding of the
concepts. The primary goal of this exercise is for you to demonstrate,
through the process of a formal write-up, that you can effectively
communicate complex ideas. We will pay close attention to the
following aspects of your write-up:
\begin{itemize}
\index{grading}
\item Content \& Accuracy (75\%) \\
    This score is based on your answers to the questions and general
    technical merit of the report.
    \begin{enumerate}
    \item The paper should be written so that someone who is
      currently taking calculus should be able to read the paper,
      understand the problem, and understand the solution
      without having seen this handout.
      {\em The paper should stand by itself}.
    \item All of the important concepts should be included and
      defined.
    \item All statements should be supported. A statement without
      justification is not acceptable.
    \item The paper should not be a disjoint collection of
      statements. We expect you to demonstrate that you know how the
      different concepts relate to one another and clearly present
      how one idea is related to and leads to another idea.
    \item All of the items listed above in the report outline should
      be included.
    \end{enumerate}

  \item Mechanics \& Presentation (25\%)  \\
    This score will be based on overall organization, neatness,
    grammar, and spelling.
    \begin{enumerate}
    \item We will count off if the grammar or spelling is so poor that
      we cannot easily understand the paper.
    \item The overall flow of the paper is important.
    \item Pay careful attention to transitions.  Transitions are
      statements that let the reader know why you are discussing what
      you are discussing.
    \end{enumerate}


\end{itemize}


\section{Oral Presentations}

\subsection{Introduction}
To help you prepare for your talk we would like to share a few helpful
tips. \index{presentations} The tips range from how to structure your
talk to some specific pointers. Please look over this document before
giving your talk and consider the hints while you are working on your
project.

\section{Mechanics}
First, we will provide a few pointers on some of the mechanics of a
talk. The mechanics are very important and are something that you will
have to practice.

\subsubsection{Starting Your Talk} Do not assume that everybody in the
room knows what your talk is going to be about! Give a brief overview
of what is happening. The overview should only take three to four
sentences and absolutely no more. \index{presentations!introduction}

Once you have let your audience know \textit{what} you will talk
about, let them know \textit{how} you will approach the problem.
Provide an outline for the talk. Explicitly state the different steps
in the problem and state who will be discussing which part of your
talk. The outline will help your audience as you make the transitions
throughout the presentation.

\subsubsection{Physical Situation}
Do not assume that everybody in the room knows what is going on!
\index{presentations|introduction} After providing an outline, provide
a description of the physical problem and state all of your
assumptions. Several of the people in the room have not seen the
problem in over a week and probably did not think real hard about it
in the first place.

\subsubsection{Who Are You Talking To?}  Face the audience when you are
talking and do not block the board. It is easy to get wrapped up with
the equations on the board, but do not forget that you are trying to
share your information with the rest of the class. Face the class and
speak in a loud clear voice. The people in the back of the room would
like to hear you.

(Tip: when I am in front of a class I find it hard to keep
the equations straight, avoid blocking what I am writing, make eye
contact with people, and project my voice all at the same time.
Sometimes, with complicated derivations, I just look at the back of
the room and try to project my voice on the back wall.)

\subsubsection{Transitions} Moving from one topic to another in the
middle of a presentation can be difficult for you, but do not forget
that it is even more difficult for your audience. Be aware of the
places in which you move from one topic to another. When this occurs,
give a brief, one or two sentence review of what you just covered and
explicitly tell them that you will be moving to the next topic.
\index{presentations!transitions}

You should provide an outline at the beginning of the talk, and you
should provide an outline for each little section in your talk as
well. As you move through the different topics in your talk, it is a
good thing to remind people of the big picture, and you should also
remind them of what you will be doing.

\subsubsection{Practice, Practice, Practice}
Practice your presentation. Practice by yourself and pay careful
attention to the timing. You should also practice with the rest of the
group. It is quite difficult to get a smooth transition when you are
passing off the speaking position to another person. Practice this!

\textit{When you practice, always keep a clock in clear view.}


\section{Mathematics}
Presenting mathematics to a group of people is a perilous game.
\index{presentations!mathematics} Some people do not want to know
details, some people do, some people just do not care, some people
care passionately, and some people are afraid. You will always have to
make decisions on how to balance your talk.

\subsubsection{Difficult Formulas} The world is a terribly complicated
place. Unfortunately, it is very rare to come across physical phenomena
that follow simple mathematical functions. To make matters worse, the
operations that are performed (derivatives, integrals, differential
equations, etc.) pervert the formulas into even more difficult forms.
When this happens, you cannot put every detail on the board and expect
your audience to follow what you are doing.

So what do you do when you have long, difficult equations that you
have to manipulate? Well, some things are better left unsaid. The
goals for a presentation are different than the goals for writing a
book. When you read a book, you want all the gory details because you
cannot ask the author questions and because you can take your time
while reading.

The goals for a presentation, on the other hand, are to give people a
more general understanding and provide them with an outline and the
details of the more interesting parts of a problem. It is okay to skip
steps, just let people know what you did. For example, it may take
five or six algebra steps to simplify an equation down into a usable
form. During a talk, do not go through every step in excruciating
detail. Just write down one or two of the steps and simply state why
you end up with the final equations. If someone does not understand a
step, they will ask!

\textit{Keep a list of all of the steps with you. Someone might ask
  you a pointed question or disagree with one of your results. Be
  prepared to justify everything that you say!}

\subsubsection{Handouts}
There are times when you cannot avoid working with very complicated
equations. For the really nasty stuff, do not write them down on the
board. Either make up a transparency or make up some pages to hand out
to the audience. \index{presentations!handouts}

The more that you hand out to the audience the better, but you should
be very careful. It can be difficult to refer to certain things on a
page that you hand out. Make sure that everything is clearly labeled.
Also, do not put items from more than one topic on a single page. This
will allow you audience to keep things separated from one another.

\textit{By the way, it is a really good idea to include an outline
  with your handouts.}


%%% Local Variables:
%%% mode: latex
%%% TeX-master: "labManual"
%%% End:
